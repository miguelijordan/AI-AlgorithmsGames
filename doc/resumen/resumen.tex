\documentclass[12pt, a4paper]{article}


\usepackage[margin=2.5cm, nohead]{geometry}
% Codificaci�n
\usepackage[utf8x]{inputenc}			% Permite el uso de caracteres acentuados.
\usepackage[T1]{fontenc}				% Letras acentuadas reales y no imitadas.
\usepackage{textcomp}					% Incluye mayor n�mero de letras con diacr�ticos.

% Idiomas
\usepackage[english, spanish]{babel} 	% Para generar documentos en idiomas distintos al ingl�s.
%\usepackage{eurosym}					% S�mbolo del euro (�).

% Espaciado
\usepackage[onehalfspacing]{setspace}	% Controla el espaciado entre l�neas.
										% \singlespacing, \onehalfspacing,
% Informaci�n
\title{ENTORNO INTERACTIVO PARA EL ESTUDIO DE ESTRATEGIAS DE I.A.
EN JUEGOS\\Resumen}
\author{José Miguel Horcas Aguilera}
\date{\today}

												
\begin{document}

	\maketitle

El objetivo del proyecto es el desarrollo de un entorno interactivo
para el estudio de algoritmos y estrategias de Inteligencia Artificial en
juegos.

Se ha desarrollado un módulo de razonamiento para los juegos, estrategias y
heurísticos.
Los juegos se representan como problemas de búsquedas entre adversarios y los
jugadores como agentes inteligentes que se desenvuelven en el espacio de estados
de los juegos.
El objetivo del agente es encontrar una estrategia ganadora.

Los juegos considerados son bipersonales, por turnos, de información perfecta,
de suma cero y deterministas; de los que se han elegido dos: el Conecta-4 y
el Go.
Se han estudiado y desarrollado los algoritmos clásicos de juegos:
minimax, alfa-beta y las tablas de transposición; además de métodos más
recientes como Monte-Carlo Tree Search.
Las estrategias contienen modificaciones prácticas como límites de tiempo para
decidir el mejor movimiento a realizar.
También se han incluido los heurísticos necesarios para evaluar los diferentes
estados de los juegos; entre ellos se encuentran un evaluador con tablas de
valor y un evaluador con red neuronal que precisan de un aprendizaje previo.

La aplicación permite al usuario jugar contra diferentes estrategias o contra
otro jugador humano, ver el desarrollo de una partida entre dos agentes
inteligentes, simular un número de partidas y analizar un estado concreto de un
juego.
En todos los casos, la aplicación proporciona estadísticas para analizar
y comparar el rendimiento y la eficacia de las estrategias.

\end{document}
